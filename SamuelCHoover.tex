\let\nofiles\relax % This is because res says to not emit aux files,
                   % but lastpage needs aux files.
\documentclass[margin,line]{res}
\usepackage[colorlinks=true]{hyperref}
\usepackage[utf8]{inputenc}
\usepackage[T1]{fontenc}
\usepackage{changepage}
\usepackage{microtype}
\usepackage{fourier-orns}
\usepackage{amsmath,amssymb}
\usepackage{lastpage}
\usepackage{fancyhdr}
\usepackage{etaremune}
\usepackage[normalem]{ulem}
\usepackage[style=iso]{datetime2}

\oddsidemargin -.5in
\evensidemargin -.5in
\voffset -25pt
%\topmargin -.2in
\headsep 25pt
\textwidth=6.0in
\textheight=8.9in
\itemsep=0in
\parsep=0in

% Headings
\pagestyle{fancy}
\lhead{Samuel C.~Hoover --- Curriculum Vitae (as of \today)}
\chead{}
\rhead{\thepage\ of \pageref*{LastPage}}
\lfoot{}
\cfoot{}
\rfoot{}
\renewcommand{\headrulewidth}{0.4pt}
%\renewcommand{\footrulewidth}{0.4pt}

% Give hyperref some metadata
\hypersetup{
  pdftitle={Samuel C. Hoover — Curriculum Vitae (CV)},
  pdfauthor={Samuel C. Hoover},
  pdfsubject={Samuel C. Hoover's academic curriculum vitae (CV)},
  pdfkeywords={machine learning, chemical engineering, data science, polymer
physics, biophysics, biomolecular condensates, intrinsically disordered
proteins}
}

\begin{document}

\newcommand{\myname}{Samuel C.~Hoover}
\newlength{\mynamewidth}
\settowidth{\mynamewidth}{\namefont\myname}

\name{\hspace*{0.5\textwidth}\hspace{-0.5\mynamewidth} \myname \vspace*{.1in}}
% On the first page, have no header.
\thispagestyle{empty}

\begin{resume}

	\section{\sc Contact Information}
%\vspace{.05in}
\href{mailto:samuel.charles.hoover@gmail.com}{samuel.charles.hoover@gmail.com}\\
\href{https://samuelhoover.github.io}{samuelhoover.github.io}\\
\href{https://www.linkedin.com/in/samuel-hoover}{linkedin.com/in/samuel-hoover}\\

%%%%%%%%%%%%%%%%%%%%%%%%%%%%%%%%%%%%%%%%%%%%

% Local Variables:
% mode: latex
% TeX-master: "SamuelCHoover.tex"
% End:


	\section{Summary}
	I am a research scientist, PPG Fellow, and Chemical Engineering Ph.D. with 6
	years of experience building data-driven models and computational tools to
	answer complex problems in the natural sciences. Proven track record of
	success through
	peer-reviewed~\href{https://scholar.google.com/citations?hl=en&user=nKKVSjwAAAAJ&vi}{publications},
	interdisciplinary internships,
	and~\href{https://github.com/samuelhoover}{open-source projects}. Adept at
	wearing many hats, working in fast-paced cross-functional teams, and
	effectively communicating difficult subjects. Seeking to applying my
	experience in computational roles.

	\section{\sc Education}
	 {\bf Ph.D., Chemical Engineering,} UMass Amherst, Amherst, MA, USA \hfill {\bf December 2024}\\
	\vspace*{-.1in}
	\begin{itemize}
		\item[ ] Committee: Murugappan Muthukumar (Chair), Peng Bai, David Hoagland, Sarah Perry
		\item[ ] Dissertation Title: {\it Study of charged macromolecule phase
		      behavior using conventional and modern modeling methods}
		\item [ ] Relevant coursework: Neural Networks, Machine Learning,
		      Computational Materials Science, Statistical Mechanics, Advanced
		      Mathematical Analysis, Transfer Process Fundamentals, Polymer Dynamics
	\end{itemize}

	{\bf B.S., Chemical Engineering,} Clarkson University, Potsdam, NY, USA \hfill {\bf May 2018}\\
	\vspace*{-.1in}
	\begin{itemize}
		\item[ ] Degree conferred with distinction.
		\item[ ] Minors: Mathematics and International \& Cross-Cultural Perspectives
		\item[ ] Relevant coursework: Mathematical Modeling, Boundary Value
		      Problems \& Fourier Series, Probability \& Statistics
	\end{itemize}

	\section{\sc Employment}
	 {\bf Research Assistant,} UMass Amherst, Amherst, MA, USA
	\hfill {\bf January 2019--December 2024}
	\begin{adjustwidth}{1.5em}{0pt}
		{\bf Lab of Prof. Murugappan Muthukumar}\\
		The overarching goal of my current doctoral research is to use simpler
		synthetic analogs to uncover the fundamental physics that govern biological
		self-assemblies like biomolecular condensates ({\bf Project 2}) and protein
		aggregations ({\bf Projects 1 \& 3}). Additionally managed group
		high-performance GPU computing cluster and static HTML webpage.
		\\
		{\bf Project 1~\textendash{} Learning the sequence effects of microphase
		separation transition}
		\begin{itemize}
			\item Created a >260k row dataset from real-world data and improved data
			      quality by identifying 5\% of sample as unreliable using
			      physics-informed filtering
			\item Developed novel molecular modeling method that uniquely captured
			      the sequence-specificity of each macromolecule
			\item Trained gradient-boosted decision trees to accurately predict
			      microphase separation transition ($R^{2} > 0.9$), 90x faster than
			      traditional methods
			\item Quantified the effects of sequence on self-assembly, found
			      second-order salt-dependent interactions on self-assembly using SHAP
			      values
		\end{itemize}
		{\bf Project 2~\textendash{} Theory of polyzwitterion-polyelectrolyte
		complexation}
		\begin{itemize}
			\item Generalized theory to account for the dynamic phase stability of
			      polyzwitterion-polyelectrolyte complexes
			\item Performed a quantitative assessment of phase stability of complexes
			      with varying molecular structures, chemistries, and solution
			      conditions, allowing for design of cargo-releasing materials
			\item Refactored group legacy free energy
			      minimization~\href{https://github.com/samuelhoover/free-energy-minimization}{script}
			      to achieve a 10x execution time speedup
		\end{itemize}
		{\bf Project 3~\textendash{} Phase behavior of polydipoles}
		\begin{itemize}
			\item Computed phase diagrams of complexation between polymers that
			      contain an electric dipole on each monomer
			\item Determined salt sensitivity with varying electrostatic dipolar
			      interaction strengths
		\end{itemize}
	\end{adjustwidth}

	% \newpage{}
	\begin{adjustwidth}{1.5em}{0pt}
		{\bf Lab of Prof. Peng Bai}\\
		The goal of this research was utilize computer vision to aid virtual
		screening of nanoporous materials like zeolites, resulted in 20,000x
		quicker materials property predictions than traditional methods.
		Additionally computed phase diagrams and forcefield parameters for small
		organic molecules.
		\\
		{\bf Project 1~\textendash{} Convolutional neural networks for virtual
		screening}
		\begin{itemize}
			\item Created
			      an~\href{https://github.com/samuelhoover/energy-grid-processing}{automated
				      pipeline} to process, analyze, and visualize over 100,000 materials in
			      HDF5 format using MATLAB
			\item Built custom
			      PyTorch~\href{https://github.com/samuelhoover/chan-ml}{framework}
			      for processing large datasets (>1 GB/sample), training, model
			      analysis, and experiment logging; ensured reproducibility and
			      reliability for 8 person research team
			\item Found simple geometric descriptors for zeolites do not provide
			      reliable predictions for their adsorption properties
		\end{itemize}
	\end{adjustwidth}
	{\bf DTMD Intern,} Triton Systems, Inc., Chelmsford, MA, USA
	\hfill {\bf June--September 2023}
	\begin{adjustwidth}{1.5em}{0pt}
		Technology \& Signal Processing Intern in the Disruptive Technology \&
		Materials Design (DTMD) group, performed electromagnetic modeling on COMSOL
		for a~\href{https://legacy.www.sbir.gov/sbirsearch/detail/2234557}{Phase II
			SBIR project} for the DHS.
		\begin{itemize}
			\item Optimized design of electromagnetic components for a handheld viral
			      detection device in collaboration with engineers
			\item Developed an easy-to-use application for product testing,
			      enabled non-technical users to perform complex finite element methods
			      calculations and estimate performance on-the-fly
			\item Supported design best practices by reviewing current literature on
			      data acquisition and signal processing for breath volatile organic
			      compound analysis
			\item Worked with key stakeholders, meeting monthly to present
			      research updates and respond to questions from financial sponsors
		\end{itemize}
	\end{adjustwidth}
	{\bf Research Assistant,} Clarkson University, Potsdam, NY, USA
	\hfill {\bf September 2017--May 2018}\\
	\\
	{\bf Global Manufacturing Tech. Intern,} SI Group, Schenectady, NY, USA
	\hfill {\bf  May--August 2017}\\
	\begin{adjustwidth}{1.5em}{0pt}
		Intern working with the Global Manufacturing Technology and Global EH\&S
		departments.
		\begin{itemize}
			\item Strengthed institutional knowledge by identifying root causes of
			      company loss events and determining impact on revenue and production
			\item Led group intern project to standardize the block flow diagram of
			      19 key assets, reduced potential errors by improving consistency and
			      clarity
			\item Implemented PI Asset Framework for real-time processing monitoring,
			      enabled quick decisions and eliminated guesswork
		\end{itemize}
	\end{adjustwidth}
	% Added to improve page breaks
	\vspace{-1em}

	\section{\sc Research Interests}
	Computational studies (machine learning, simulation, theory) of synthetic and
	biological polymeric systems. One major theme is to elucidate the physical
	mechanisms underpinning protein aggregations and self-assemblies that give
	rise to neurodegeneration. Another theme is to use complex coacervates as
	models for biomolecular condensates to further understand the spatiotemporal
	organization within the cell.

	\section{\sc Honors and Awards}
	 {\bf PPG Fellowship,} PPG Industries, Inc. \hfill {\bf 2024}\\
	\\
	{\bf Teaching Assistant Award,} University of Massachusetts Amherst \hfill {\bf Fall 2022}\\
	\\
	{\bf Clarkson Scholarship,} Clarkson University \hfill {\bf Fall 2014--Spring 2018}\\
	\\
	{\bf Dean's List,} Clarkson University \hfill {\bf Fall 2014--Fall 2017}\\

	% Added to improve page breaks
	\vspace{-1em}

	\section{\sc Teaching Experience}
	 {\bf Teaching Assistant}, University of Massachusetts Amherst
	\vspace*{.05in}
	\begin{itemize}
		\item[ ] CHEM-ENG 401, Senior Laboratory \hfill {\bf Falls 2022--2023}
		\item[ ] CHEM-ENG 338, Separation Processes \hfill {\bf Spring 2022}
		\item[ ] CHEM-ENG 446, Process Control \hfill {\bf Fall 2021}
	\end{itemize}
	%\newpage{}
	{\bf Tutor}, Clarkson University
	\vspace*{.05in}
	\begin{itemize}
		\item[ ] STAT 383, Probability and Statistics \hfill {\bf Spring 2018}
		\item[ ] CH 370, Transfer Process Fundamentals \hfill {\bf Fall 2017}
	\end{itemize}
	{\bf Teaching Assistant}, Clarkson University
	\vspace*{.05in}
	\begin{itemize}
		\item[ ] CH 370, Transfer Process Fundamentals \hfill {\bf Fall 2017}
		\item[ ] ES 100, Intro to Engineering Use of Computers \hfill {\bf Spring 2016}
	\end{itemize}
	{\bf Senior Teaching Assistant}, Clarkson University
	\vspace*{.05in}
	\begin{itemize}
		\item[ ] ES 100, Intro to Engineering Use of Computers \hfill {\bf Spring 2017}
	\end{itemize}

	\section{\sc Professional Membership, Activities, Outreach, and Service}
	 {\bf University of Massachusetts Graduate Student Senate}
	\vspace*{.05in}
	\begin{itemize}
		\item[ ] Senator \hfill {\bf September 2019--May 2021}
	\end{itemize}
	{\bf Order of the Engineer}
	\vspace*{.05in}
	\begin{itemize}
		\item[ ] Member \hfill {\bf 2018--Present}
	\end{itemize}
	{\bf Omega Chi Epsilon}
	\vspace*{.05in}
	\begin{itemize}
		\item[ ] Delta Chapter President \hfill {\bf 2017--2018}
		\item[ ] Member \hfill {\bf 2016--Present}
	\end{itemize}
	{\bf American Institute of Chemical Engineers}
	\vspace*{.05in}
	\begin{itemize}
		\item[ ] Clarkson University ChemE Car Senior Advisor \hfill {\bf 2017--2018}
		\item[ ] Clarkson University ChemE Car President \hfill {\bf 2016--2017}
		\item[ ] Clarkson University ChemE Car Treasurer \hfill {\bf 2015--2016}
		\item[ ] Member \hfill {\bf 2014--Present}
	\end{itemize}

	{\bf Outreach}
	\vspace*{.05in}
	\begin{itemize}

		\item[] AIChE Eckhardt Northeast Student Regional Conference volunteer
		      \hfill {\bf March 30, 2019} \\
		      \hspace*{1em} Hosted ChemE Jeopardy Competition

		\item[] Extended Day STEM Peer Educator
		      \hfill {\bf September 2017--May 2018} \\
		      \hspace*{1em} After school activities to teach STEM to local middle
		      school students on a weekly basis

	\end{itemize}

	\section{\sc Skills}
	%{\bf Languages---}%
	Proficient in Python, Rust, C/C++, Bash, MATLAB, and SQL. Proficient at *nix and
	Windows based systems and cloud-based high-performance computing. Markup
	languages: \LaTeX, HTML, CSS.

		% {\bf Operating systems---}%
		% Mac OS, Linux/*nix.

		{\bf Methods---}%
	Polymer physics, biophysics, molecular dynamics, machine learning,
	statistical modeling, data engineering, explainable machine learning, deep
	learning, and computational biology.


		{\bf Software---}%
	Most contributions can be found at \url{https://github.com/samuelhoover}.
	Proficient in machine learning and numerical toolkits like PyTorch,
	scikit-learn, scikit-image, NumPy, SciPy, pandas, PostgreSQL, XGBoost, and SHAP.
	Experience with coarse-grained and atomistic molecular dynamics packages like
	GROMACS, LAMMPS, PyMOL, Avogadro, Schr\"odinger, and VMD.
	Experience with development tools like Git, Docker, and AWS.
	Experience developing machine learning and deep learning pipelines for
	physical science research.
	Experience with finite element analysis with COMSOL.
	Proficient with visualization tools like Matplotlib, seaborn, and Inkscape.

	% Twiddle this for spacing
	% \newpage

	\ifx\nopubs\undefined
		\newcommand{\arxiv}[1]{[\href{http://arxiv.org/abs/#1}{arXiv:#1}]}
% Citation counts last updated 2023-08-06
\def\zero{0}
\def\one{1}
\newcommand{\citeCount}[1]{%
	\def\val{#1}
	\ifx\val\zero%
	\else%
		\ifx\val\one%
			(1~citation)%
		\else%
			(#1~citations)%
		\fi%
	\fi}

% Comment out to show, uncomment to hide
\renewcommand{\citeCount}[1]{}

\newcounter{numPubs}
\newcounter{pubCounter}

\setcounter{numPubs}{3}
\setcounter{pubCounter}{\value{numPubs}}

\section{\sc Publications in Progress}
\begin{etaremune}[start=\value{pubCounter}]
	\item
	      {\bf Hoover,~S.~C.},
	Li,~S.~-F.,
	Muthukumar, M.
	(2024)
	{\it Learning the sequence effect on the microphase separation transition of charged heteropolymers}.
	\setcounter{pubCounter}{\value{enumi}}
\end{etaremune}

%%%%%%%%%%
%%%%%%%%%%
%%% As of 2024-09-06
%%%%%%%%%%
%%%%%%%%%%
\newif\ifshowpubsummary
%%% Comment out the next line to omit the publication summary
% \showpubsummarytrue
%%%
\ifshowpubsummary
	\section{\sc Publication Summary}
	 {\bf h-index ---}%
	As of 2024-06-03: 61 (according to Google Scholar), or 53 (according
	to INSPIRE).  Both include collaboration papers.

		{\bf Top five cited ---}%
	Excluding LIGO/Virgo collaboration papers.
	%Citation counts from Google Scholar.
	\begin{enumerate}
		\item
		      Berti, E., (5 authors), {\bf Stein,~L.~C.}, (46 more authors)
		      (2015)
		      {\it Testing General Relativity with Present and Future
				      Astrophysical Observations},
		      \href{http://dx.doi.org/10.1088/0264-9381/32/24/243001}{Class. Quantum Grav. {\bf 32} 243001}
		      \arxiv{1501.07274}.
		      \citeCount{1321}
		\item
		      Barack,~L., {\it et al.}
		      (2019)
		      {\it Black holes, gravitational waves and fundamental physics: a roadmap},
		      \href{https://doi.org/10.1088/1361-6382/ab0587}{Class.~Quantum Grav.~{\bf 36} 143001}
		      \arxiv{1806.05195}.
		      \citeCount{669}
		\item
		      Boyle,~M., {\it et al.} ({\bf LCS} is corresponding author)
		      (2019)
		      {\it The SXS Collaboration catalog of binary black hole simulations},
		      \href{https://doi.org/10.1088/1361-6382/ab34e2}{Class.~Quantum Grav.~{\bf 36} 195006}
		      \arxiv{1904.04831}.
		      \citeCount{320}
		\item
		      Varma,~V, {\it et al.}
		      (2019)
		      {\it Surrogate models for precessing binary black hole simulations with
				      unequal masses},
		      \href{https://doi.org/10.1103/PhysRevResearch.1.033015}{Phys.~Rev.~Research~{\bf 1},~033015}
		      \arxiv{1905.09300}.
		      \citeCount{302}
		\item
		      Yunes,~N., {\bf Stein,~L.~C.}
		      (2011),
		      {\it Nonspinning black holes in alternative theories of gravity},
		      \href{http://dx.doi.org/10.1103/PhysRevD.83.104002}{Phys.~Rev.~D~{\bf 83}~104002}
		      \arxiv{1101.2921}.
		      \citeCount{243}
	\end{enumerate}
\else% don't show pub summary
\fi

\renewcommand{\citeCount}[1]{}

% \section{\sc Submitted Publications}
% %\addtocounter{pubCounter}{-1}
% \begin{etaremune}[start=\value{pubCounter}]
% 	\item
% 	Magaña~Zertuche,~L.,
% 	{\bf Stein,~L.~C.},
% 	{\it et al.},
% 	(2024)
% 	{\it High-Precision Ringdown Surrogate Model for Non-Precessing Binary Black Holes},
% 	\arxiv{2408.05300}.
% 	\citeCount{0}
% 	\setcounter{pubCounter}{\value{enumi}}
% \end{etaremune}
%
% \section{\sc Accepted Publications}
% \addtocounter{pubCounter}{-1}
% \begin{etaremune}[start=\value{pubCounter}]
% \item
%   {\bf Stein,~L.~C.},
%   (2024)
%   {\it Can a radiation gauge be horizon-locking?},
%   \arxiv{2404.10113}.
%   \citeCount{0}
%   \setcounter{pubCounter}{\value{enumi}}
% \end{etaremune}

% \section{\sc Collaboration Publications}
% From 2008--2012, I was coauthor on 34 refereed LIGO and/or LIGO/Virgo
% collaboration publications. I only list short author-list publications below.

\section{\sc Refereed Publications}
\addtocounter{pubCounter}{-1}
\begin{etaremune}[start=\value{pubCounter}]

	\item
	      {\bf Hoover,~S.~C.},
	Margossian,~K.~O.,
	Muthukumar, M.
	(2024)
	{\it Theory and quantitative assessment of pH-responsive polyzwitterion-polyelectrolyte complexation},
	\href{https://doi.org/10.1039/D4SM00575A}{Soft Matter~{\bf 20},~7199-7213}.

	\item
	Liu,~Y.,
	Perez,~G.,
	Cheng,~Z.,
	Sun,~A.,
	{\bf Hoover,~S.~C.},
	Fan,~W.,
	Maji,~S.,
	Bai,~P.
	(2023)
	{\it ZeoNet: 3D convolutional neural networks for predicting adsorption in nanoporous zeolites},
	\href{https://doi.org/10.1039/D3TA01911J}{J. Mater. Chem. A~{\bf 11},~17570-17580}.

	\setcounter{pubCounter}{\value{enumi}}
\end{etaremune}

%%%%%%%%%%%%%%%%%%%%%%%%%%%%%%%%%%%%%%%%%%%%

% Local Variables:
% mode: latex
% TeX-master: "SamuelCHoover.tex"
% End:

	\else
		%
	\fi

	%\newcommand{\playsymbol}{\framebox[1.3\width]{$\blacktriangleright$}}
\newcommand{\playsymbol}{$\blacktriangleright$}
\section{\sc Presentations and Conferences}
\begin{etaremune}
	\item
	UMass Amherst Chemical Engineering G.R.A.S.S. talk
	\hfill{}
	October 2023
	\item
	Center for UMass / Industry Research on Polymers poster session
	\hfill{}
	October 2023
	\item
	Center for UMass / Industry Research on Polymers poster session
	\hfill{}
	May 2023
	\item
	UMass Amherst Chemical Engineering Graduate Open House poster session
	\hfill{}
	March 2023
	\item
	Nanopore Sequencing: From Genomes to Proteomes poster session
	\hfill{}
	May 2022
	\item
	Center for UMass / Industry Research on Polymers poster session
	\hfill{}
	May 2022
	\item
	NHGRI Advanced Genomic Technology Development virtual meeting
	\hfill{}
	May 2021
\end{etaremune}

%%%%%%%%%%%%%%%%%%%%%%%%%%%%%%%%%%%%%%%%%%%%

% Local Variables:
% mode: latex
% TeX-master: "SamuelCHoover.tex"
% End:


	% \newpage{}

	\section{\sc References}
	\parbox{\textwidth}{%
		Available upon request.
	}

\end{resume}
\end{document}

% Local Variables:
% mode: latex
% End:
